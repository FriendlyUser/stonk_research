
% Save this as tutorial.tex for the lwarp package tutorial.

\documentclass{book}

\usepackage{iftex}

% --- LOAD FONT SELECTION AND ENCODING BEFORE LOADING LWARP ---

\ifPDFTeX
\usepackage{lmodern}            % pdflatex or dvi latex
\usepackage[T1]{fontenc}
\usepackage[utf8]{inputenc}
\else
\usepackage{fontspec}           % XeLaTeX or LuaLaTeX
\fi

% --- LWARP IS LOADED NEXT ---
\usepackage[
%   HomeHTMLFilename=index,     % Filename of the homepage.
%   HTMLFilename={node-},       % Filename prefix of other pages.
%   IndexLanguage=english,      % Language for xindy index, glossary.
%   latexmk,                    % Use latexmk to compile.
%   OSWindows,                  % Force Windows. (Usually automatic.)
%   mathjax,                    % Use MathJax to display math.
]{lwarp}
% \boolfalse{FileSectionNames}  % If false, numbers the files.

% --- LOAD PDFLATEX MATH FONTS HERE ---

% --- OTHER PACKAGES ARE LOADED AFTER LWARP ---
\usepackage{makeidx} \makeindex
\usepackage{xcolor}             % (Demonstration purposes only.)
\usepackage{hyperref,cleveref}  % LOAD THESE LAST!

% --- LATEX AND HTML CUSTOMIZATION ---
\title{The Lwarp Tutorial}
\author{Some Author}
\setcounter{tocdepth}{2}        % Include subsections in the \TOC.
\setcounter{secnumdepth}{2}     % Number down to subsections.
\setcounter{FileDepth}{1}       % Split \HTML\ files at sections
\booltrue{CombineHigherDepths}  % Combine parts/chapters/sections
\setcounter{SideTOCDepth}{1}    % Include subsections in the side\TOC
\HTMLTitle{Webpage Title}       % Overrides \title for the web page.
\HTMLAuthor{Some Author}        % Sets the HTML meta author tag.
\HTMLLanguage{en-US}            % Sets the HTML meta language.
\HTMLDescription{A description.}% Sets the HTML meta description.
\HTMLFirstPageTop{Name and \fbox{HOMEPAGE LOGO}}
\HTMLPageTop{\fbox{LOGO}}
\HTMLPageBottom{Contact Information and Copyright}
\CSSFilename{lwarp_sagebrush.css}

\begin{document}

\maketitle                      % Or titlepage/titlingpage environment.

% An article abstract would go here.

\tableofcontents                % MUST BE BEFORE THE FIRST SECTION BREAK!
\listoffigures

\chapter{Stonk Picks}

Blah Blah Blah


\section{EarthRenew}

- pretty heavy position here, about 10\% of portfolio.

\subsection{April 2022}

\textbf{Bullish Case}

ERTH.CN

Aiming for a 10 bagger
\begin{itemize}
    \item ESG, fertilizer shortage
    \item company executing well
    \item expanding to new facilities
    \item expecting revenue of 24 million where market cap is around 24 million when purchased
    \item regenerative fertilizer with premium for organic, competitive, meets a lot of what the government of Canada wants to invest in, likely to get government subsidies
    \item high insider ownership
\end{itemize}

\textbf{Bearish Case}

\begin{itemize}
    \item Fertilizer prices go down as high prices result in low prices eventually
    \item Russia war ends within the war (doubt sanctions will go down right away)
    \item Failed to be profitable in Q4, need to raise more capital (unlikely, but possible)
    \item fail to uplist to senior exchanges and raise capital needed for expansion
    \item 
\end{itemize}


WATR.V 

If they figure out green hydrogen, 100 times gain. Easily, I think its a 100 bagger, but we will have to see.


RET.V
purchased on April, 23, 2022.

Bullish case
\begin{itemize}
    \item successfully out of bankruptcy, covid reopening, 3 profitable quarters, low pe
    \item share holder equity about the value of the market cap
    \item ecommerce intiatives (if success should be an easy double or triple)
    \item high levels of immigration should increase demand
\end{itemize}

Bearish case

\begin{itemize}
    \item economic collapse because of idiot powell, probably just stonk collapse
    \item failure to capitalize on opportunities and changing consumer demand, etc ....
\end{itemize}

\begin{warpprint}   % For print output ...
\cleardoublepage    % ... a common method to place index entry into TOC.
\phantomsection
\addcontentsline{toc}{chapter}{\indexname}
\end{warpprint}
\ForceHTMLPage      % HTML index will be on its own page.
\ForceHTMLTOC       % HTML index will have its own toc entry.
\printindex

\end{document}